\begin{backchapter}
\begin{frontmatter}
\chapter{Introduction to MATLAB}\label{app2}
\makechaptertitle
\end{frontmatter}

\def\thesection{\thechapter.\arabic{section}}

Throughout this book, we shall use MATLAB to plot functions, to
multiply matrices, and to solve problems involving systems of linear
equations. MATLAB has an extensive library of commands that makes it
possible to solve such problems in a simple straightforward
way.\vspace*{15pt}


\section{Creating a Vector}\label{app2:sec1}

We begin by describing how one creates a row vector. The command line in MATLAB begins with the symbol $>>$.  This symbol is  written by the MATLAB program itself as a prompt at the beginning of each line. An interactive session intended to create a row vector $x$ with elements, 0.0, 0.25, 0.50, 0.75, and 1.0, could be
\begin{verbatim}
>> x = [0 0.25 0.50 0.75 1.0]
\end{verbatim}
%\pagebreak


\noindent MATLAB would respond to this command by writing

\begin{verbatim}
x  =
         0 0.2500 0.500 0.7500 1.00
\end{verbatim}

\noindent Notice that MATLAB  writes the output of the command on the screen. One can suppress this output  by typing a semicolon (;) at the end of the line.

\end{backchapter}
