\subsection{Microeconomic Evidence}\label{subsec:microEvidence}

\subsubsection{Background}
In discussions elsewhere we mention efforts to calibrate parameters of specific epidemiological models to particular kinds of data.  But our definition of an epidemiological process as one in which social interactions affect people's beliefs and consequent economic behaviors opens a broad field for empirical work.  Here, we summarize a literature that collects evidence in ways not specifically targeted to estimating the parameters of a particular epidemiological model of an economic phenomenon.

The gathering of evidence in a less structural way has many useful purposes as an exploratory step before the construction of formal model. In principle, such work could answer questions like
\begin{quote}
    \normalfont
\begin{enumerate}
    \item What are the characteristics of source and recipient of the infection
    \item Under what conditions and through which media do communications take place
    \item What kinds of information/expectations are more infectious?
    \item Are economic choices truly affected by identifiable socially transmitted beliefs
    \end{enumerate}
\end{quote}

Among the reasons epidemiological modeling has been slow to spread among economists, one is surely that every one of these questions has been difficult to answer directly using traditional data sources available to economists.

But the burgeoning ``social network'' data that have begun to be used in economic research offer rich opportunities for profoundly improving our ability to measure such things.

\subsubsection{Older Papers Using ``Neighbors''}

In the absence of direct evidence about the nature and frequency of social contacts between people, the economics literature has naturally relied upon plausible proxies.   For instance,  \href{http://www.columbia.edu/~hh2679/ThyNeighborJF.pdf}{\cite{hong2005thy}} found that fund managers tend to buy similar stocks to other fund managers in the same city. \href{https://github.com/iworld1991/EpiExp/blob/master/Literature/hvide2015social.pdf}{\cite{hvide2015social}} found that stock market investment decisions of individuals are positively correlated with those of coworkers.  \href{https://www.jstor.org/stable/10.1086/592415}{\cite{cohen2008small}} shows that fund managers place larger bets and perform better on socially connected (similar education backgrounds) firms than on other firms.  In addition, social interaction also affects stock market participation and stock choices, as shown in  \href{https://github.com/iworld1991/EpiExp/blob/master/Literature/hong2004social.pdf}{\cite{hong2004social}}, \href{https://onlinelibrary.wiley.com/doi/abs/10.1111/j.1540-6261.2008.01364.x}{\cite{brown2008neighbors}}, and \href{https://github.com/iworld1991/EpiExp/blob/master/Literature/ivkovic2007information.pdf}{\cite{ivkovic2007information}}. % hong2004social:  more social households more likely to invest in the stock market using data from HRS;  brown2008neighbors:  one more likely owns stocks in higher ownership communities, instrumenting the community ownership by nonnative residents' ownership.
 In the context of housing market investment,
 \href{https://www.aeaweb.org/articles?id=10.1257/aer.20171611&from=f}{\cite{bayer2021speculative}} shows that novice investors entered the market after seeing investing activity in the same neighborhoods.

\subsubsection{Social Networks}

In a world with ubiquitous social networks, the set of people who can influence economic expectations is no longer limited to peers who are physically nearby.  \href{https://www.journals.uchicago.edu/doi/abs/10.1086/700073}{\cite{bailey2018economic}} show, essentially, that people who happen randomly to have social-network friends in distant cities where home prices have crashed are more pessimistic about their local housing market, and less likely to buy, than people whose remote friends happen to live in places where house prices have not crashed.  (The paper is of course careful to control for every imaginable confound.)

\href{https://www.aeaweb.org/articles?id=10.1257/aer.102.4.1414}{\cite{iyer2012understanding}} study the dynamics of an actual bank run using high-frequency data on deposit withdrawals among persons connected in a social network.   \href{https://www.aeaweb.org/articles?id=10.1257/aer.90.5.1110}{\cite{kelly2000market}} showed that depositors who learned from acquaintances about the bad news regarding bank were the first to close bank account.

%Word-of-mouth communications might be even particularly important in spreading the information in fraudulent contagion and speculative investment activities. For instance, \href{https://github.com/iworld1991/EpiExp/blob/master/Literature/rantala2019investment.pdf}{\cite{rantala2019investment}} provides direct evidence on diffusion of investment ideas among a large Ponzi scheme. \href{https://files.fisher.osu.edu/department-finance/public/information_networks_evidence_from_illegal_insider_trading_tips.pdf}{\cite{ahern2017information}}shed light on the information flow of illegal insider trading among strong social ties.

%\subsubsection{News Media}
%Social communication not only takes the form of conversations within social circles but also via mass media, especially given the content disseminated publicly from the media is ultimately generated by individual people and subject to subjective bias. Furthermore, it takes time for the news to spread through the entire population. Therefore, the literature on how news media affect individual investment and market prices bear important implications for the epidemiological models as well. For example,  \href{http://faculty.haas.berkeley.edu/odean/papers\%20current\%20versions/allthatglitters_rfs_2008.pdf}{	\cite{tetlock_giving_2007}} shows that non-fundamental sentiment from financial news drives trading volumes of the relevant stocks. \href{https://www.researchgate.net/publication/227465410_Journalists_and_the_Stock_Market}{\cite{dougal2012journalists}} shows that WSJ columnists' persistent “bullishness” or “bearishness” explains excess returns of the stock market. \href{https://www.public.asu.edu/~dsosyura/ResearchPapers/Rumor\%20Has\%20It\%20--\%20Sensationalism\%20in\%20Financial\%20Media.pdf}{\cite{ahern2015rumor}}:  inaccurate news reports have impacts on asset prices. In a similar spirit,  \href{https://github.com/iworld1991/EpiExp/blob/master/Literature/hvide2015social.pdf}{\cite{ahern2015rumor}} identified rumors of merging news (unrealized later) of public firms lead to abnormal trading and stock price movements. \href{http://faculty.haas.berkeley.edu/odean/papers\%20current\%20versions/allthatglitters_rfs_2008.pdf}{\cite{barber_all_2008}} found that individual investors are more likely to buy attention-grabbing stocks.
 \href{https://www.stern.nyu.edu/sites/default/files/assets/documents/con_040497.pdf}{\cite{soo_quantifying_2015}}:  housing media sentiment has significant predictive power for future house prices.

\subsubsection{How Does Shiller's `Narrative Approach' Fit In?}\label{narrativeApproach}

Shiller has speculated that the driving force in aggregate fluctuations, both for asset markets and for macroeconomies, is the varying prevalence of alternative `narratives' that people believe capture the key `story' of how the economy is working (his earliest statement of this view seems to be \href{https://www.jstor.org/stable/2117915}{\cite{shiller1995conversation}}).

He has returned to this theme more recently, and our opening quote from him makes it clear that he thinks narratives spread by ``going viral.''  See \citep{shiller2017narrative,shiller_narrative_2019} for more extended treatments.

There are several challenges for turning Shiller's view into a quantitative modeling tool.  One is the difficulty of identifying the alternative narratives competing at any time in the population.
\href{https://github.com/iworld1991/EpiExp/blob/master/Literature/shiller2020popular.pdf}{\cite{shiller2020popular}} made an initial effort at this.  By reading over historical news archives and internet search records, he identified six major economic narratives that have circulated during the economic expansion since 2009, including ``Great Depression,'' ``secular stagnation,'' ``sustainability,'' ``housing bubble,'' ``strong economy,'' and ``save more.''

\cite{larsen2019business} have recently taken up the formidable challenge of quantifying media narratives, deriving virality indexes, and conducting
Granger causality tests to determine the extent to which the viral narratives have predictive power, in the U.S., Japan, and Europe.

It will be interesting to watch the development of this literature as tools like Natural Language Processing and various forms of Artificial Intelligence develop.
It is not beyond imagining that at some point it might be possible to train AI tools to comb through the vast amount of information contained in social network communications to identify economic narratives, and to measure the ways in which they spread.  At that point it might be possible to construct a thoroughly satisfactory epidemiological model of Shiller's narrative theory of economic fluctuations -- and to see how effective it is.  But that date is still some distance in the future.



 \subsubsection{Summary}
 Three important patterns regarding epidemiological modeling emerge from all the studies above.  First, none of them is compatible with the identical-Rational-Expectations model -- if expectations were identical, there would be nothing to transmit. Therefore, modelers need to specify not only the mechanisms of the infection but also what is the content of the infection.   Second, different information/opinions spreading across the population may differ in terms of their infectiousness depending on many factors such as newsworthiness, salience, and sentiment. This suggests that modelers may need to allow the possibility for certain features of the information content to endogenously affect its infection rate and recovery rate.  Third, the structure of social networks -- who you interact with, and how frequently -- can affect the process of transmission and potentially the equilibrium outcome.  The implication for the modelers here is that the detailed structure of the social connections via which the ideas spread matters for the aggregate dynamics.
 If Democrats and Republicans do not assort randomly with each other in their social connections, or do not accord equal weight to opinions of persons of the opposite party, it is not hard to see how striking results like those in the \cite{meeuwis2018belief} paper could arise.